\documentclass{article}
\def\inputGnumericTable{}
\usepackage[latin1]{inputenc}
\usepackage{fullpage}
\usepackage{color}
\usepackage{array}
\usepackage{longtable}
\usepackage{calc}
\usepackage{multirow}
\usepackage{hhline}
\usepackage{ifthen}

% --- Additional packages required for Math, Images, and Lists ---
\usepackage{amsmath}
\usepackage{amsfonts}
\usepackage{amssymb}
\usepackage{graphicx}
\usepackage{enumitem} 
\usepackage{float}
\usepackage[top=1in, bottom=1in, left=0.8in, right=0.8in]{geometry}

% Header/Footer setup
\usepackage{fancyhdr}
\pagestyle{fancy}
\fancyhf{}
\fancyhead[R]{Math Exercises}
\fancyfoot[C]{\thepage}
\renewcommand{\headrulewidth}{0pt}

\begin{document}

% ==========================================
% HEADER SECTION
% ==========================================

% --- Logo and Student Info ---
\noindent
\begin{minipage}[t]{0.5\textwidth}
    \vspace{0pt} 
    % Ensure Logo.png is in the same folder
    \includegraphics[width=5cm]{Logo.png} 
\end{minipage}%
\begin{minipage}[t]{0.5\textwidth}
    \vspace{0pt}
    \begin{description}[font=\bfseries, style=multiline, leftmargin=3.5cm]
        \item[Name:] sivaram
        \item[ID:] cometfwc039
        \item[Date of joining:] 20/01/2026
    \end{description}
\end{minipage}

\vspace{1cm}
\hrule
\vspace{1cm}

% ==========================================
% EXERCISE 4.1 CONTENT
% ==========================================

\begin{center}
    {\Large \textbf{EXERCISE 4.1}}
\end{center}

\noindent Evaluate the determinants in Exercises 1 and 2.

\begin{enumerate}[label=\textbf{\arabic*.}, leftmargin=*, itemsep=1.5em]

    % Question 1
    \item $\begin{vmatrix} 2 & 4 \\ -5 & -1 \end{vmatrix}$

    % Question 2
    \item 
    \begin{enumerate}[label=(\roman*), itemsep=1em]
        \item $\begin{vmatrix} \cos \theta & -\sin \theta \\ \sin \theta & \cos \theta \end{vmatrix}$
        \hspace{2cm}
        (ii) $\begin{vmatrix} x^2 - x + 1 & x - 1 \\ x + 1 & x + 1 \end{vmatrix}$
    \end{enumerate}

    % Question 3
    \item If $A = \begin{bmatrix} 1 & 2 \\ 4 & 2 \end{bmatrix}$, then show that $|2A| = 4|A|$.

    % Question 4
    \item If $A = \begin{bmatrix} 1 & 0 & 1 \\ 0 & 1 & 2 \\ 0 & 0 & 4 \end{bmatrix}$, then show that $|3A| = 27|A|$.

    % Question 5
    \item Evaluate the determinants:
    \begin{enumerate}[label=(\roman*), itemsep=1em]
        \item $\begin{vmatrix} 3 & -1 & -2 \\ 0 & 0 & -1 \\ 3 & -5 & 0 \end{vmatrix}$
        \hspace{2cm}
        (ii) $\begin{vmatrix} 3 & -4 & 5 \\ 1 & 1 & -2 \\ 2 & 3 & 1 \end{vmatrix}$
        
        \vspace{0.5cm}
        
        \item $\begin{vmatrix} 0 & 1 & 2 \\ -1 & 0 & -3 \\ -2 & 3 & 0 \end{vmatrix}$
        \hspace{2cm}
        (iv) $\begin{vmatrix} 2 & -1 & -2 \\ 0 & 2 & -1 \\ 3 & -5 & 0 \end{vmatrix}$
    \end{enumerate}

    % Question 6
    \item If $A = \begin{bmatrix} 1 & 1 & -2 \\ 2 & 1 & -3 \\ 5 & 4 & -9 \end{bmatrix}$, find $|A|$.

    % Question 7
    \item Find values of $x$, if
    \begin{enumerate}[label=(\roman*), itemsep=1em]
        \item $\begin{vmatrix} 2 & 4 \\ 5 & 1 \end{vmatrix} = \begin{vmatrix} 2x & 4 \\ 6 & x \end{vmatrix}$
        \hspace{2cm}
        (ii) $\begin{vmatrix} 2 & 3 \\ 4 & 5 \end{vmatrix} = \begin{vmatrix} x & 3 \\ 2x & 5 \end{vmatrix}$
    \end{enumerate}

    % Question 8
    \item If $\begin{vmatrix} x & 2 \\ 18 & x \end{vmatrix} = \begin{vmatrix} 6 & 2 \\ 18 & 6 \end{vmatrix}$, then $x$ is equal to
    \begin{enumerate}[label=(\Alph*), itemsep=0.5em, listparindent=1.5cm]
        \item 6 \hspace{2cm} (B) $\pm 6$ \hspace{2cm} (C) $-6$ \hspace{2cm} (D) 0
    \end{enumerate}

\end{enumerate}

\newpage

% ==========================================
% EXERCISE 4.4 CONTENT
% ==========================================

\begin{center}
    {\Large \textbf{EXERCISE 4.4}}
\end{center}

\noindent Write Minors and Cofactors of the elements of following determinants:

\begin{enumerate}[label=\textbf{\arabic*.}, leftmargin=*, itemsep=1.5em]

    % Question 1
    \item 
    \begin{enumerate}[label=(\roman*)]
        \item $\begin{vmatrix} 2 & -4 \\ 0 & 3 \end{vmatrix}$
        \hspace{3cm}
        (ii) $\begin{vmatrix} a & c \\ b & d \end{vmatrix}$
    \end{enumerate}

    % Question 2
    \item 
    \begin{enumerate}[label=(\roman*)]
        \item $\begin{vmatrix} 1 & 0 & 0 \\ 0 & 1 & 0 \\ 0 & 0 & 1 \end{vmatrix}$
        \hspace{3cm}
        (ii) $\begin{vmatrix} 1 & 0 & 4 \\ 3 & 5 & -1 \\ 0 & 1 & 2 \end{vmatrix}$
    \end{enumerate}

    % Question 3
    \item Using Cofactors of elements of second row, evaluate $\Delta = \begin{vmatrix} 5 & 3 & 8 \\ 2 & 0 & 1 \\ 1 & 2 & 3 \end{vmatrix}$.

    % Question 4
    \item Using Cofactors of elements of third column, evaluate $\Delta = \begin{vmatrix} 1 & x & yz \\ 1 & y & zx \\ 1 & z & xy \end{vmatrix}$.

    % Question 5
    \item If $\Delta = \begin{vmatrix} a_{11} & a_{12} & a_{13} \\ a_{21} & a_{22} & a_{23} \\ a_{31} & a_{32} & a_{33} \end{vmatrix}$ and $A_{ij}$ is Cofactors of $a_{ij}$, then value of $\Delta$ is given by
    \begin{enumerate}[label=(\Alph*), itemsep=0.8em]
        \item $a_{11} A_{31} + a_{12} A_{32} + a_{13} A_{33}$
        \item $a_{11} A_{11} + a_{12} A_{21} + a_{13} A_{31}$
        \item $a_{21} A_{11} + a_{22} A_{12} + a_{23} A_{13}$
        \item $a_{11} A_{11} + a_{21} A_{21} + a_{31} A_{31}$
    \end{enumerate}

\end{enumerate}

\newpage

% ==========================================
% EXERCISE 4.5 CONTENT
% ==========================================

\begin{center}
    {\Large \textbf{EXERCISE 4.5}}
\end{center}

\noindent Find adjoint of each of the matrices in Exercises 1 and 2.

\begin{enumerate}[label=\textbf{\arabic*.}, leftmargin=*, itemsep=2em]

    % Question 1 & 2
    \item $\begin{bmatrix} 1 & 2 \\ 3 & 4 \end{bmatrix}$
    \hspace{4cm}
    \textbf{2.} $\begin{bmatrix} 1 & -1 & 2 \\ 2 & 3 & 5 \\ -2 & 0 & 1 \end{bmatrix}$
    \setcounter{enumi}{2} % Manually set counter for next item

    \vspace{0.5cm}
    \noindent Verify $A (\text{adj } A) = (\text{adj } A) A = |A| I$ in Exercises 3 and 4.
    \vspace{0.5cm}

    % Question 3 & 4
    \item $\begin{bmatrix} 2 & 3 \\ -4 & -6 \end{bmatrix}$
    \hspace{4cm}
    \textbf{4.} $\begin{bmatrix} 1 & -1 & 2 \\ 3 & 0 & -2 \\ 1 & 0 & 3 \end{bmatrix}$
    \setcounter{enumi}{4}

    \vspace{0.5cm}
    \noindent Find the inverse of each of the matrices (if it exists) given in Exercises 5 to 11.
    \vspace{0.5cm}

    % Question 5 & 6 & 7
    \item $\begin{bmatrix} 2 & -2 \\ 4 & 3 \end{bmatrix}$
    \hfill
    \textbf{6.} $\begin{bmatrix} -1 & 5 \\ -3 & 2 \end{bmatrix}$
    \hfill
    \textbf{7.} $\begin{bmatrix} 1 & 2 & 3 \\ 0 & 2 & 4 \\ 0 & 0 & 5 \end{bmatrix}$
    \setcounter{enumi}{7}

    \vspace{0.5cm}

    % Question 8 & 9 & 10
    \item $\begin{bmatrix} 1 & 0 & 0 \\ 3 & 3 & 0 \\ 5 & 2 & -1 \end{bmatrix}$
    \hfill
    \textbf{9.} $\begin{bmatrix} 2 & 1 & 3 \\ 4 & -1 & 0 \\ -7 & 2 & 1 \end{bmatrix}$
    \hfill
    \textbf{10.} $\begin{bmatrix} 1 & -1 & 2 \\ 0 & 2 & -3 \\ 3 & -2 & 4 \end{bmatrix}$
    \setcounter{enumi}{10}

    \vspace{0.5cm}

    % Question 11
    \item $\begin{bmatrix} 1 & 0 & 0 \\ 0 & \cos\alpha & \sin\alpha \\ 0 & \sin\alpha & -\cos\alpha \end{bmatrix}$

    % Question 12
    \item Let $A = \begin{bmatrix} 3 & 7 \\ 2 & 5 \end{bmatrix}$ and $B = \begin{bmatrix} 6 & 8 \\ 7 & 9 \end{bmatrix}$. Verify that $(AB)^{-1} = B^{-1} A^{-1}$.

    % Question 13
    \item If $A = \begin{bmatrix} 3 & 1 \\ -1 & 2 \end{bmatrix}$, show that $A^2 - 5A + 7I = O$. Hence find $A^{-1}$.

    % Question 14
    \item For the matrix $A = \begin{bmatrix} 3 & 2 \\ 1 & 1 \end{bmatrix}$, find the numbers $a$ and $b$ such that $A^2 + aA + bI = O$.

    % Question 15
    \item For the matrix $A = \begin{bmatrix} 1 & 1 & 1 \\ 1 & 2 & -3 \\ 2 & -1 & 3 \end{bmatrix}$, Show that $A^3 - 6A^2 + 5A + 11I = O$. Hence, find $A^{-1}$.

    % Question 16
    \item If $A = \begin{bmatrix} 2 & -1 & 1 \\ -1 & 2 & -1 \\ 1 & -1 & 2 \end{bmatrix}$, Verify that $A^3 - 6A^2 + 9A - 4I = O$ and hence find $A^{-1}$.

    % Question 17
    \item Let A be a nonsingular square matrix of order $3 \times 3$. Then $|\text{adj } A|$ is equal to
    \begin{enumerate}[label=(\Alph*), itemsep=0.5em, listparindent=1.5cm, labelsep=1em]
        \item $|A|$ \hfill (B) $|A|^2$ \hfill (C) $|A|^3$ \hfill (D) $3|A|$
    \end{enumerate}

    % Question 18
    \item If A is an invertible matrix of order 2, then det($A^{-1}$) is equal to
    \begin{enumerate}[label=(\Alph*), itemsep=0.5em, listparindent=1.5cm, labelsep=1em]
        \item det (A) \hfill (B) $\frac{1}{\text{det (A)}}$ \hfill (C) 1 \hfill (D) 0
    \end{enumerate}

\end{enumerate}

\end{document}
